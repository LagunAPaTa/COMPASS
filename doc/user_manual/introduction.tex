\chapter{Introduction}

The OpenATS COMPASS (\textbf{Comp}liance \textbf{Ass}essment) tool aims at providing a generalized framework for ATC surveillance data inspection and analysis. \\

Please \textbf{note} that COMPASS was previously called ATSDB, and includes all functionality previously present in ATSDB. \\

This document has its focus on interaction and working procedures required to make use of the existing
functionality. In this introduction, feature highlights and ackknowledgments are listed, followed by a brief summary of important aspects of  COMPASS in \nameref{sec:key_concepts}. \\

In the section \nameref{sec:installation}, prerequisites are listed and installation instructions are given. In the section \nameref{sec:test_data}, a few comments about how to obtain/use test data are given. \\

In the larger section \nameref{sec:tasks}, the steps are described to run the application, create a database or access an existing one, import and processing data beforte the start of the management GUI. How configured tasks can be executed automatically is described in \nameref{sec:command_line}. \\

How data can be loaded is described in the section \nameref{sec:management}, re-visiting saved points of interest is described in \nameref{sec:view_points}. Inspection of loaded data using the two existing views is described in the sections \nameref{sec:listbox_view} and \nameref{sec:osg_view}. \\

The evaluation feature is described in the section \nameref{sec:eval}, describing how requirement-based standards can be adapted/defined and compliance to said standards can be assessed.

In the section \nameref{sec:troubleshooting} details about reported isses are collected, as well as details on how to report new issues. \\

In the last section \nameref{sec:appendix} additional details are given. In the section \nameref{sec:appendix_utils} some information is given how data can be manually imported into COMPASS. In the final section \nameref{sec:appendix_licensing} information is given about under what conditions COMPASS can be used and what 
libraries with what licences are used in the background.

\pagebreak

\subfile{intro_feature_highlights}

\subfile{intro_display}

\subfile{intro_general_aspects}

\subfile{intro_acknowledgements}

\subfile{intro_key_concepts}


